\def\year{2015}
%File: formatting-instruction.tex
\documentclass[letterpaper]{article}
\usepackage{aaai}
\usepackage{times}
\usepackage{helvet}
\usepackage{courier}
\usepackage{graphicx}
\frenchspacing
\setlength{\pdfpagewidth}{8.5in}
\setlength{\pdfpageheight}{11in}
\pdfinfo{
/Title (Affective Motivational Collaboration Theory)
/Author (Put All Your Authors Here, Separated by Commas)}
\setcounter{secnumdepth}{0}  
 \begin{document}
% The file aaai.sty is the style file for AAAI Press 
% proceedings, working notes, and technical reports.
%
\title{Affective Motivational Collaboration Theory}
\author{Mohammad Shayganfar, Charles Rich, Candace L. Sidner\\
Worcester Polytechnic Institute\\
Comoputer Science Department\\
100 Institute Rd, Worcester, MA 01609\\
}
\maketitle
\begin{abstract}
\begin{quote}
We investigate the mutual influence of affective and collaboration processes in
a cognitive theory to support the interaction between humans and robots or
virtual agents. We will develop new algorithms for these processes, as well as a
new overall computational model for implementing collaborative robots and
agents. We build primarily on the \textit{cognitive appraisal} theory of
emotions \cite{gratch:domain-independent} and the \textit{SharedPlans} theory
\cite{grosz:plans-discourse} of collaboration to investigate the structure,
fundamental processes and functions of emotions in a collaboration context. As
part of this work, we also address a deficiency in existing cognitive models by
accounting for the influence of motivation on collaborative behaviors, such as
overcoming an impasse. This motivation mechanism uses the results of cognitive
appraisal to dynamically form new beliefs and intentions related to the
collaboration structure.
\end{quote}
\end{abstract}

\noindent Ronald De Sousa in The Rationality of Emotion
\cite{sousa:rationality-emotion} makes a good case for the claim that humans are
capable of rationality largely because they are creatures with emotions. The
idea of having robots or other intelligent agents living in a human environment
has been a persistent dream from science fiction books to artificial
intelligence and robotics laboratories. However, there are many challenges in
achieving collaboration between robots and humans in the same environment. Some
of these challenges involve physical requirements, some involve cognitive
requirements, and some involve social requirements. Thus far, there has been an
emphasis on the design of robots to deal with the physical requirements. Many
researchers are also working on the cognitive requirements, inspired by a
diverse set of disciplines. As time passes, there has been an increasing
recognition of the importance of the social requirements.

\section{Motivation}

One aspect of the sociability of robots and agents is their ability to
collaborate with humans in the same environment. Therefore, it is important to
understand what makes a collaboration effective. One's cognitive processes and
the ability to understand the collaborative environment impact the effectiveness
of a collaboration. Examples of cognitive capabilities that support the
effectiveness of collaboration include: a) perceiving one's own internal states
and b) communicating them, c) coordinating personal and group behaviors, d)
identifying self and mutual interests, e) recognizing the accountability of
private and shared goals, f) selecting appropriate actions with respect to
events, and g) engaging others in collaboration.

We want to investigate the cognitive processes involved in a collaboration in
the context of a cognitive architecture. There are several well-developed
cognitive architectures, e.g., Soar \cite{laird:soar} and ACT-R
\cite{anderson:act-r}, each with different approaches to defining the basic
cognitive and perceptual operations. There have also been efforts to integrate
affect into these architectures
\cite{dancy:actR-physiology-affect,marinier:behavior-emotion}. In general,
however, these cognitive architectures do not focus on processes to specifically
produce emotion-regulated goal-driven collaborative behaviors. At the same time,
existing collaboration theories, e.g., SharedPlans \cite{grosz:plans-discourse}
theory, focus on describing the structure of a collaboration in terms of
fundamental mental states, e.g., mutual beliefs or joint intentions. However,
they do not describe the associated processes, their relationships, and
influences on each other. \textit{Affective Motivational Collaboration Theory}
deals with all of the major processes, including affective and motivational
processes, having an impact on the collaboration structure. This theory is
informed by research in psychology and artificial intelligence. Our
contribution, generally speaking, is to synthesize prior work on motivation,
appraisal and collaboration, and thus to provide a new theory which describes
the prominent emotion-regulated goal-driven phenomena in a dyadic collaboration.

\section{Collaboration and Emotion}

Collaboration is a coordinated activity in which the participants work jointly
to satisfy a shared goal \cite{grosz:plans-discourse}. There are many important
unanswered questions about the involvement of an individual's cognitive
abilities during collaboration. Some of these questions are related to the
dynamics of collaboration, as well as the underlying mechanisms and processes.
For instance, a general mechanism has yet to be developed that allows an agent
to initiate proactive collaborative behaviors when it faces a blocked task.
There is also lack of a general mechanism that, in the event of a task failure,
allows an agent to consider the human's anticipated mental states and emotions,
while managing its own internal goals as well as the collaboration's shared
goal. There are also other questions about the components involved in these
processes at the cognitive level, such as the processes that are involved for
evaluative, regulatory or motivative purposes as internal processes of the
agent. There has also not been enough attention on the processes that are
involved to maintain the social aspects of a collaboration.

Emotions have a key role in influencing the cognitive processes involved in
social interaction and collaboration. Emotion processing and decision-making are
integral aspects of daily life and maintain their prominence during social
interaction and collaboration. However, researchers' understanding of the
interaction between emotions and collaborative behaviors is limited. We believe
that the evaluative role of emotions as a part of cognitive processes helps an
agent to perform appropriate behaviors during a collaboration. To work jointly
in a coordinated activity, participants (collaborators) act based on their own
understanding of the world and the anticipated mental states of the counterpart;
this understanding is reflected in their collaborative behaviors. Emotions are
pivotal in the collaboration context, since their regulatory and motivative
roles enhance an individual's autonomy and adaptation as well as his/her
coordination and communication competencies in a dynamic, uncertain and
resource-limited environment.

\begin{figure}[tbh]
  \centering
  \includegraphics[width=0.474\textwidth]{figure/theory-general.png}
  \caption{Computational framework based on \textit{Affective Motivational
  Collaboration Theory} (primary influences between mechanisms).}
  \label{fig:cpm}
\end{figure}

\section{Affective Motivational Collaboration Theory}

We build \textit{Affective Motivational Collaboration Theory} on
\textit{SharedPlans} theory of collaboration \cite{grosz:plans-discourse}
and \textit{cognitive appraisal} theory of emotions
\cite{gratch:domain-independent}. \textit{Affective Motivational Collaboration
Theory} is about the interpretation and prediction of the observable behaviors
in a dyadic collaborative interaction. The theory focuses on the processes
regulated by emotional states. It aims to explain both rapid emotional reactions
to events as well as slower, more deliberative responses. The observable
behaviors represent the outcome of reactive and deliberative processes related
to the interpretation of the self's relationship to the collaborative
environment. The reactive and deliberative processes are triggered by two types
of events: \textit{external} events, such as the other's \textit{utterances} and
\textit{primitive actions}, and \textit{internal} events, comprising changes in
the self's mental states, such as belief formation and emotional changes.
\textit{Affective Motivational Collaboration Theory} explains how emotions
regulate the underlying processes in the occurrence of these events during
collaboration. This theory will elucidate the role of motives as goal-driven
emotion-regulated constructs with which an agent can form new beliefs and
intentions to cope with internal and external events.

\textit{Affective Motivational Collaboration Theory} will explain the functions
of emotions in a dyadic collaboration and show how affective mechanisms can
coordinate social interactions by anticipating other's emotions, beliefs and
intentions. Our focus will be on mechanisms depicted in Figure \ref{fig:cpm} as
mental processes along with the mental states. The \textit{Mental States} (see
Figure \ref{fig:cpm}) includes self's (robot's) beliefs, intentions, motives,
goals and emotion instances as well as the anticipated Mantal States of the
other's (human's). The \textit{Collaboration} mechanism maintains constraints on
actions. These constraints include constraints on task states and on the
ordering of tasks. The \textit{Collaboration} mechanism also provides processes
to update and monitor the shared plan. The \textit{Appraisal} mechanism is
responsible for evaluating changes in the self's Mental States, the anticipated
Mental States of the other, and the state of the collaboration environment. The
\textit{Coping} mechanism provides the self with different coping strategies
associated with changes in the self's mental states with respect to the state of
the collaboration. The \textit{Motivation} mechanism operates whenever the self
a) requires a new motive to overcome an internal impasse in an ongoing task, or
b) wants to provide an external motive to the other when the other faces a
problem in a task. The \textit{Theory of Mind} mechanism is the mechanism of
inferring a model of the other's anticipated Mental States. The self will
progressively update this model during the collaboration. We will develop and
evaluate the \textit{Affective Motivational Collaboration Theory} and the
associated computational model in a simulated (two-participant) human-robot
collaboration.

\bibliography{mshayganfar.bib}
\bibliographystyle{aaai}

\end{document}
